El método de las Potencias se usa frecuentemente para hallar el autovalor dominante de una matriz. Se llama método de las potencias, porque implícitamente se construye a través de las potencias de $A$.

Sean los autovalores $\lambda_1, \lambda_2, \dots, \lambda_n$ de $A$ tal que
\[ |\lambda_1| > |\lambda_2| \geq |\lambda_3| \geq \dots \geq |\lambda_n| \]
esto es, $\lambda_1$ es el autovalor dominante de $A$. Sea $v_1$ el correspondiente autovector asociado a $\lambda_1$. Si $max(g)$ denota el elemento de módulo máximo del vector $g$. Asumamos que $A$ es diagonalizable.
\vspace{2mm}

\begin{algorithm}[H]
\caption{Método de las Potencias}
\DontPrintSemicolon
Paso 1. Elegir $x_0$\;
Paso 2. \For{$k = 1, 2, 3, \dots$ \textbf{hasta} convergencia}{
    $\hat{x}_k = A x_{k-1}$\;
    $x_k = \frac{\hat{x}_k}{\max \hat{x}_k}$\;
}
\end{algorithm}
A continuación se presenta la implementación del algoritmo:

\lstinputlisting[language=Python]{../scripts/eigen-v/m-potencias.py}


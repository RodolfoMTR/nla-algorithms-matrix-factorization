La importancia de las matrices de Householder es que pueden crear ceros en las columnas de una matriz por debajo de la entrada de la diagonal principal de $A$.
\begin{deff}{}{}
Una matriz de la forma
\[
H = I - \frac{2uu^\top}{u^\top u}
\]
donde $u$ es un vector no nulo, es llamada \textbf{matriz de Householder}.
\end{deff}
\begin{lema}{}{}
Dado un vector no nulo $x \neq e_1$, existe una matriz de Householder $H$ tal que $Hx$ es múltiplo de $e_1$
\end{lema}
\begin{prueba}
Definamos $H = I - \frac{2uu^\top}{u^\top u}$ con $u = x + sign(x_1) \|x\|_2 e_1$. Entonces, multiplicando, se puede ver que $Hx$ es un múltiplo de $e_1$
\end{prueba}
\begin{algorithm}[H]
\DontPrintSemicolon
\SetKwInOut{Input}{Entrada}\SetKwInOut{Output}{Salida}

\Input{Vector $u \in \mathbb{R}^n$ que define la matriz de Householder $H$, vector $x \in \mathbb{R}^n$.}
\Output{Vector $x$ sobrescrito con el producto $Hx$.}

\BlankLine
\tcp{Paso 1: Calcular el factor de escala}
$\beta \leftarrow \frac{2}{u^\top u}$ \;

\tcp{Paso 2: Calcular el producto escalar $s = u^\top x$}
$s \leftarrow \sum_{i=1}^{n} u_i x_i$ \;

\tcp{Paso 3: Actualizar $\beta$ con el producto escalar}
$\beta \leftarrow \beta \cdot s$ \;

\tcp{Paso 4: Actualizar el vector x}
\For{$i = 1$ \KwTo $n$}{
    $x_i \leftarrow x_i - \beta u_i$ \;
}

\caption{Producto de un vector con una matriz de Householder ($Hx$)}
\end{algorithm}

A continuación se presenta la implementación del algoritmo:

\lstinputlisting[language=Python]{../scripts/factorization/householder.py}


\documentclass{article}
\usepackage{amsmath,amsfonts,amssymb,amsthm}
\usepackage{mathrsfs}
\usepackage[spanish]{babel}
\usepackage{listings}
\usepackage{xcolor}
\usepackage[ruled,vlined]{algorithm2e}
% Configuración para que el código se vea limpio
\lstset{
    language=Python,
    basicstyle=\ttfamily\small,
    keywordstyle=\color{blue},
    commentstyle=\color{green!60!black},
    stringstyle=\color{orange},
    numbers=left,
    numberstyle=\tiny,
    frame=single,
    breaklines=true
}
\newcommand{\R}{\mathbb R}
\newcommand{\A}{\mathcal A}
\newcommand{\B}{\mathcal B}
\newcommand{\Rc}{\mathcal R}
\newcommand{\N}{\mathbb N}
\newcommand{\T}{\mathscr{T}}
\usepackage[utf8]{inputenc}
\setlength{\parindent}{0cm}%sin identación
\usepackage{graphicx}
\usepackage{float}
\usepackage[left=2.54cm, right=2.54cm,top=2.54cm,bottom=2.54cm,a4paper,heightrounded]{geometry}
\usepackage{tcolorbox}%for theorems
\tcbuselibrary{theorems}%for theorems
%for theorems
\newtcbtheorem[number within=section]{theo}{Teorema}%
{colback=green!5,colframe=green!35!black,fonttitle=\bfseries}{th}
\newtcbtheorem[number within=section]{prop}{Proposición}%
{colback=green!5,colframe=green!35!black,fonttitle=\bfseries}{prop}

\newtcbtheorem[number within=section]{cor}{Corolario}%
{colback=magenta!5,colframe=magenta!35!black,fonttitle=\bfseries}{cor}

\newtcbtheorem[number within=section]{lema}{Lema}%
{colback=yellow!5,colframe=yellow!35!black,fonttitle=\bfseries}{lem}

\newtcbtheorem[number within=section]{deff}{Definición}%
{colback=blue!5,colframe=blue!70!black,fonttitle=\bfseries}{deff}

\newtheorem*{remark}{Observación}
\newtheorem{exercise}{Ejercicio}
\newenvironment{prueba}{\paragraph{Demostración:}\quad\\}{\hfill$\blacksquare$}
%------titulo
\title{Implementación de algoritmos de álgebra lineal numérica}
\author{Rodolfo M. Turpo R.}
\date{12 de diciembre de 2025}
%-----
\begin{document}
\maketitle
\tableofcontents
\newpage

\section{Matriz de Householder}
La importancia de las matrices de Householder es que pueden crear ceros en las columnas de una matriz por debajo de la entrada de la diagonal principal de $A$.
\begin{deff}{}{}
Una matriz de la forma
\[
H = I - \frac{2uu^\top}{u^\top u}
\]
donde $u$ es un vector no nulo, es llamada \textbf{matriz de Householder}.
\end{deff}
\begin{lema}{}{}
Dado un vector no nulo $x \neq e_1$, existe una matriz de Householder $H$ tal que $Hx$ es múltiplo de $e_1$
\end{lema}
\begin{prueba}
Definamos $H = I - \frac{2uu^\top}{u^\top u}$ con $u = x + sign(x_1) \|x\|_2 e_1$. Entonces, multiplicando, se puede ver que $Hx$ es un múltiplo de $e_1$
\end{prueba}
\begin{algorithm}[H]
\DontPrintSemicolon
\SetKwInOut{Input}{Entrada}\SetKwInOut{Output}{Salida}

\Input{Vector $u \in \mathbb{R}^n$ que define la matriz de Householder $H$, vector $x \in \mathbb{R}^n$.}
\Output{Vector $x$ sobrescrito con el producto $Hx$.}

\BlankLine
\tcp{Paso 1: Calcular el factor de escala}
$\beta \leftarrow \frac{2}{u^\top u}$ \;

\tcp{Paso 2: Calcular el producto escalar $s = u^\top x$}
$s \leftarrow \sum_{i=1}^{n} u_i x_i$ \;

\tcp{Paso 3: Actualizar $\beta$ con el producto escalar}
$\beta \leftarrow \beta \cdot s$ \;

\tcp{Paso 4: Actualizar el vector x}
\For{$i = 1$ \KwTo $n$}{
    $x_i \leftarrow x_i - \beta u_i$ \;
}

\caption{Producto de un vector con una matriz de Householder ($Hx$)}
\end{algorithm}

A continuación se presenta la implementación del algoritmo:

\lstinputlisting[language=Python]{../scripts/factorization/householder.py}


\section{Factorización QR}
A continuación se presenta la implementación del algoritmo:

\lstinputlisting[language=Python]{../scripts/factorization/f-qr.py}


\section{Factorización de cholesky}
Para
\[ A = HH^\top \]
\[
\begin{bmatrix}
a_{11} & a_{12} & \cdots & a_{1n} \\
a_{21} & a_{22} & \cdots & a_{2n} \\
\vdots & \vdots & \ddots & \vdots \\
a_{n1} & a_{n2} & \cdots & a_{nn}
\end{bmatrix}
=
\begin{bmatrix}
h_{11} & 0 & \cdots & 0 \\
h_{21} & h_{22} & \cdots & 0 \\
\vdots & \vdots & \ddots & \vdots \\
h_{n1} & h_{n2} & \cdots & h_{nn}
\end{bmatrix}
\begin{bmatrix}
h_{11} & h_{21} & \cdots & h_{n1} \\
0 & h_{22} & \cdots & h_{n2} \\
\vdots & \vdots & \ddots & \vdots \\
0 & 0 & \cdots & h_{nn}
\end{bmatrix}
\]

Tenemos:
\begin{itemize}
    \item $h_{11} = \sqrt{a_{11}}$
    \item $h_{i1} = \frac{a_{i1}}{h_{11}}, \quad i = 2, 3, \dots, n$
    \item $\sum_{k=1}^{i} h_{ik}^2 = a_{ii}$
    \item $a_{ij} = \sum_{k=1}^{j} h_{ik}h_{kj}, \quad j < i$
\end{itemize}
\begin{algorithm}[H]
\DontPrintSemicolon
\SetKwInOut{Input}{Entrada}\SetKwInOut{Output}{Salida}

\Input{Una matriz $A \in \mathbb{R}^{n \times n}$ simétrica definida positiva.}
\Output{Factor de Cholesky $H$ (matriz triangular superior).}

\BlankLine
\tcp{El algoritmo construye H fila por fila y sobrescribe la parte superior de A}
\For{$k = 1, 2, \dots, n$}{
    
    \For{$i = 1, 2, \dots, k-1$}{
        $a_{ki} \equiv h_{ki} = \frac{1}{h_{ii}} \left( a_{ki} - \sum_{j=1}^{i-1} h_{ij}h_{kj} \right)$ \;
    }
    
    \BlankLine
    \tcp{Cálculo del elemento de la diagonal}
    $a_{kk} \equiv h_{kk} = \sqrt{a_{kk} - \sum_{j=1}^{k-1} h_{kj}^2}$ \;
}

\caption{Algoritmo de Factorización de Cholesky}
\end{algorithm}

A continuación se presenta la implementación del algoritmo:

\lstinputlisting[language=Python]{../scripts/factorization/f-cholesky.py}


\section{minimos cuadrados}
\section{Método de las potencias}
El método de las Potencias se usa frecuentemente para hallar el autovalor dominante de una matriz. Se llama método de las potencias, porque implícitamente se construye a través de las potencias de $A$.

Sean los autovalores $\lambda_1, \lambda_2, \dots, \lambda_n$ de $A$ tal que
\[ |\lambda_1| > |\lambda_2| \geq |\lambda_3| \geq \dots \geq |\lambda_n| \]
esto es, $\lambda_1$ es el autovalor dominante de $A$. Sea $v_1$ el correspondiente autovector asociado a $\lambda_1$. Si $max(g)$ denota el elemento de módulo máximo del vector $g$. Asumamos que $A$ es diagonalizable.
\vspace{2mm}

\begin{algorithm}[H]
\caption{Método de las Potencias}
\DontPrintSemicolon
Paso 1. Elegir $x_0$\;
Paso 2. \For{$k = 1, 2, 3, \dots$ \textbf{hasta} convergencia}{
    $\hat{x}_k = A x_{k-1}$\;
    $x_k = \frac{\hat{x}_k}{\max \hat{x}_k}$\;
}
\end{algorithm}
A continuación se presenta la implementación del algoritmo:

\lstinputlisting[language=Python]{../scripts/eigen-v/m-potencias.py}


\end{document}
